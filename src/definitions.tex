\begin{description}
\item[Amis] Ensemble de personnes avec lesquelles un orga souhaite travailler 
\item[BdE] Bureau des Élèves de l'INSA de Lyon
\item[Catégorie d'Orgas] Orgas bénéficiant le même niveau de comfiance. (par exemple : orgas Hard, orgas BdE, orgas Soft)
\item[Commentaire] Texte horodaté et signé pouvant être ajouté à une tâche ou à un groupe de tâches.
\item[Crénau] Plage horaire correspondant à l'exécution d'une tâche.
\item[Équipe] Au sein d'une association, groupe d'orgas ayant des responsabilités touchant à un même domaine. (par exemple : Équipe Logistique, Équipe Animations)
\item[Lieu] Position géographique où des tâches doivent être effectuées.
\item[Manifestation] Événement organisé par une équipe ou une assocuation du BdE.
\item[Matériel] Objet ou véhicule nécessaire à la manifestation. Il peut être individuellement identifiable (Un transpalette, une clé) ou non (des barrières, des talkies-walkies)
\item[Orga] Personne effectuant des tâches lors d'une manifestation.
\item[Orga Humain] Lors d'une manifestation, personne chargée de la gestion des ressources humaines.
\item[Orga IF] Équipe chargée du système d'information du BdE. Désigne également un membre de cette équipe.
\item[Planning] Ensemble des crénaux devant être effectués par un orga. (Où, quand et pour faire quoi)
\item[Priorité] Importance d'une tâche. Une tâche à la priorité faible ou nulle pourra avoir des crénaux non assignés à des orgas.
\item[Tâche] Travail devant être effectué par un orga.
\item[Validation] Processus permettant l'élaboration des tâches par plusieurs orgas, et l'insertion dans le système avec l'accord d'un chef d'équipe.
\item[Voyage] Action de transporter, à l'aide d'un véhicule, du matériel et/ou des orgas d'un lieu à un autre.


 \end{description}