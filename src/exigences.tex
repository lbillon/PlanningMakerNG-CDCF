
\section{Orgas}



\fonction{Inscription}
Les orgas peuvent insérer leurs propres informations dans le système lors de l'inscription, et pouvoir les modifier plus tard.

Les orgas peuvent effectuer ces opérations à l'aide d'une interface web, accessible en ligne.

\fonction{Authentification}
Les orgas peuvent accéder à leurs données et les modifier de manière sécurisée, qu'ils soient ou non étudiants à l'INSA de Lyon.

Il est préférable que les étudiants de l'INSA s'authentifient à l'aide du système CAS.

\fonction{Disponibilités}
Les orgas peuvent déclarer plusieurs plages horaires pendant lesquelles ils peuvent travailler.

\fonction{Amis}
Les orgas peuvent indiquer les personnes avec lesquelles ils souhaitent travailler. Ces souhaits doivent être respectés dans la mesure du possible.

\subsection{Lettres d'excuses aux départements}
\fonction{Génération et impression}
Le logiciel permet de générer et d'imprimer des lettres d'excuses adressées aux directeurs des départements.

\fonction{Personnalisation}
Les éléments suivants peuvent être ajoutés et modifiés aux lettres d'excuse : 
\begin{itemize}
 \item Nom, adresse, et logotype de l'association
\item Nom et signature du président de l'association.
\item Département et année de l'orga excusé
\item Nom et adresse du destinataire
\item Date
\end{itemize}


\fonction{Administration des orgas}
L' \oh{} peut modifier les informations concernant un orga.


\subsection{Catégories d'orgas}
\fonction{Création}
L'\oh{} peut créer un nombre illimité de catégories d'orgas.

\fonction{Assignation des orgas}
Les orgas peuvent se déclarer comme membre d'une catégorie d'orgas, cette affectation devient active après validation de l'\oh{}. Un orga peut appartenir à plusieurs catégories, certaines peuvent être cachées.

\section{Tâches}

\begin{figure}[h!t]
\centering
\includegraphics[width=\textwidth]{process_taches.png}
\label{fig:ptaches}
\caption{Processus de création et de validation d'un groupe de tâches.}
\end{figure}
\clearpage

\fonction{Création}
Un orga peut créer un nombre illimité de tâches.

\fonction{Validation}
Une tâche peut être validée par le chef de l'équipe à laquelle se rapporte la tâche ou par un \oh{}.

\fonction{Commentaires}
Un nombre illimité de commentaires peuvent être ajoutés à une tâche, et ce par différents orgas.

\fonction{Importation/Exportation}
Les tâches (et notamment leurs description) peuvent être exportées vers et importées depuis un format de fichier standard.


\subsection{Groupe de tâches}
\fonction{Inclusion de tâches}
Une tâche est nécessairement incluse dans un groupe de tâches.
\fonction{Création}
Un orga peut créer un groupe de tâches directement lors de la création d'une nouvelle tâche.

\fonction{Commentaires}
Un nombre illimité de commentaires peuvent être ajoutés à un groupe de tâches, et ce par différents orgas.

\section{Lieux}
\fonction{Création}
Un orga peut créer un lieu directement lors de la création d'une nouvelle tâche.

\section{Matériel}

\fonction{Création}
Un orga peut créer un matériel directement lors de la création d'une nouvelle tâche.

\fonction{Besoins en Matériel}
L'\oh{} peut générer, visualiser et imprimer la quantité totale de matériel nécessaire, heure par heure.

\fonction{Planning}
L'\oh{} peut générer, visualiser et imprimer le planning d'un matériel.

\section{Créneaux}
\begin{figure}[h!t]
\centering
\includegraphics[]{tache-crenaux.png}
\label{fig:ptaches}
\caption{Décomposition d'une tâche en crénaux.}
\end{figure}



\fonction{Création}
Le logiciel peut créer automatiquement des crénaux pour une tâche.

\fonction{Planification automatique}
Le logiciel peut assigner automatiquement des crénaux aux orgas.

\fonction{Résultats à produire}
Le logiciel peut générer et afficher les états suivants : 

\begin{itemize}
\item Nombre de crénaux assignés, restant à assigner et orgas disponibles en fonction du temps.
\item Liste des crénaux, triés et groupés indifférament par : 	\begin{itemize}
								  \item Orga
								  \item Équipe
								  \item Horaire
								 \end{itemize}
\item Liste des orgas, triés et groupés indifférament par :  	\begin{itemize}
								  \item Catégorie
								  \item Équipe
								  \item Département
								 \end{itemize}
                                     

\end{itemize}



\fonction{Affichage à l'écran}
Le logiciel permet un affichage intéractif des résultats produits.

\begin{figure}[h!t]
\centering
\includegraphics[width=\textwidth]{interactivite.png}
\label{fig:ptaches}
Un clic sur une information ouvre le dialogue indiqué par la flèche en pointillés..
\caption{Démonstration de l'interactivité}
\end{figure}
\clearpage


\fonction{Impression}
Le logiciel peut produire et imprimer les résultats produits sous forme adaptée à l'impression.