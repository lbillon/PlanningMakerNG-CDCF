\section{Contexte}
Tout au long de l'année, les associations et les équipes du BdE organisent un grand nombre de manifestations. Lors de leur préparation, la gestion des ressources humaines est un
 élément clé, qui prend souvent un temps considérable, et qui se révèle complexe.


\section{Objectifs}
Afin de gérer simplement et efficacement la répartition des différentes tâches entre les organisateurs, et permettre la génération de plannings pour ces organisateurs, nous devons
développer une nouvelle version du logiciel PlanningMaker, en prenant compte des nouveaux besoins de utilisateurs.

\subsection{Détail des prestations}
L'équipe Orga If du BdE réalisera :
\begin{enumerate}
 \item L'étude préalable au développement du logiciel, qui comprend notamment le recueil des besoins des futurs utilisateurs.
\item Le développement du logiciel.
\item L'encadrement du passage au nouveau système, ce qui comprend notamment : \begin{itemize}
                                                                                \item L'écriture d'une documentation expliquant le fonctionnement du logiciel.
\item La formation des utilisateurs.
                                                                               \end{itemize}
\item La conduite des tests.
\item Le déploiement du logiciel au sein des associations concernées.

\end{enumerate}


\section{Solutions existantes}
Actuellement, un logiciel permettant la gestion des ressources humaines est utilisé sur plusieurs manifestations. Cependant, de nombreuses fonctionnalités sont peu adaptées aux besoins actuels des associations.
\subsection{Axes d'évolution}
\begin{description}
 \item[Ergonomie] De nombreuses fonctions sont difficilement accessibles, et certaines sont difficiles à comprendre.
\item[Champ d'utilisation] Le logiciel doit pouvoir être utilisé quelle que soit la taille de la manifestations, tout en restant efficace.
\item[Efficacité Technique] Actuellement, l'assignation des orgas aux crénaux se fait de manière manuelle et très peu assistée.
\end{description}