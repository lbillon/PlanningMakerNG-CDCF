\section{Fiabilité}

\fonction{Sauvegarde des données}
Le logiciel permet de sauvegarder et de charger des données présentes sur tous types de supports.

\fonction{Entrées Utilisateurs}
Le logiciel teste la validité de toutes les données fournies en entrée.

\section{Portabilité}
\fonction{Environnement}
Le logiciel est compatible avec les systèmes d'exploitation et architectures suivantes : 
\begin{itemize}
\item Linux i386
\item MacOS X 10.5.8 et supérieur
\item Windows XP et supérieur
\end{itemize}

\fonction{Mobilité}
Le logiciel est en mesure de fonctionner sur un ordinateur portable non relié à Internet ou à un autre réseau. 
Les fonctions suivantes : inscription en ligne des orgas, et accès multi-utilisateurs ne seront pas assurées dans cette configuration.

\fonction{Installabilité}
Le logiciel est en mesure de créer lui-même les tables et bases de données externes dont il peut avoir besoin.


\section{Sécurité}
\fonction{Gestion des droits}
Le logiciel intégre un système de gestion de droits basé sur la catégorie des orgas.


\section{Variabilité}
\fonction{Taille de la manifestation}
Le logiciel peut être utilisé efficacement sur toutes tailles de manifestations susceptibles d'être organisées sur l'INSA de Lyon.

\fonction{Association}
Les informations relative à l'association sont modifiables directement dans le programme.

\fonction{Paramétrage spécifique}
Les paramètres de l'application (par exemple: nombre d'heures de sommeil minimal/orga, pour le contrôle d'erreurs) sont modifiables directement dans le programme.

\section{Ergonomie}
\fonction{Internationalisation}
L'interface utilisateur est présentée en Français. Les unités utilisées sont celles du système métrique. Les heures seront données dans le fuseau horaire français (CET et CEST)

\fonction{Multi-Ultilisateurs}
Plusieurs personnes peuvent travailler en même temps sur la même source de données.


